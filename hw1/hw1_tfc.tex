\documentclass[12pt]{article}
\usepackage[framed,numbered,autolinebreaks,useliterate]{mcode}
\usepackage[margin=1.0in]{geometry}
\title{HW \#1}
\author{
        Travis Collins \\
	traviscollins@wpi.edu \\
	ECE 578 Crytography and Data Security
}
\date{\today}
\usepackage{titling}
\setlength{\droptitle}{-1.0in}


\begin{document}
\maketitle

\section{Ciphertext 1}
\textbf{Methodology:}\\
The first process was to determine what cipher was used, which was done by utilizing the attacks demonstrated in class.  Starting with the easy attacks and working my way upward the attack that produced results was the Affine cipher.  This was discovered by the brute force method, which is quite reasonable since there are only \[ 26 * 12 = 312 \] possible permutations.  This was done by loading the text into MATLAB for visual inspection and determining the values for \textbf{a} and \textbf{b}: \[D(c)=a^{-1}(c-b)\]

\noindent
\textbf{Translated:}\\
Alice was beginning to get very tired of sitting by her sister on the
bank, and of having nothing to do: once or twice she had peeped into the
book her sister was reading, but it had no pictures or conversations in
it, ’and what is the use of a book,’ thought Alice ’without pictures or
conversation?’

So she was considering in her own mind (as well as she could, for the
hot day made her feel very sleepy and stupid), whether the pleasure
of making a daisy-chain would be worth the trouble of getting up and
picking the daisies, when suddenly a White Rabbit with pink eyes ran
close by her.

There was nothing so VERY remarkable in that; nor did Alice think it so
VERY much out of the way to hear the Rabbit say to itself, ’Oh dear!
Oh dear! I shall be late!’ (when she thought it over afterwards, it
occurred to her that she ought to have wondered at this, but at the time
it all seemed quite natural); but when the Rabbit actually TOOK A WATCH
OUT OF ITS WAISTCOAT-POCKET, and looked at it, and then hurried on,
Alice started to her feet, for it flashed across her mind that she had
never before seen a rabbit with either a waistcoat-pocket, or a watch
to take out of it, and burning with curiosity, she ran across the field
after it, and fortunately was just in time to see it pop down a large
rabbit-hole under the hedge.

In another moment down went Alice after it, never once considering how
in the world she was to get out again.

The rabbit-hole went straight on like a tunnel for some way, and then
dipped suddenly down, so suddenly that Alice had not a moment to think
about stopping herself before she found herself falling down a very deep
well.



\section{Ciphertext 2}
\textbf{Methodology:}\\
I used the same process as before to determine the cipher type for the encrypted text.  It reduced quickly to a Vigenere Cipher, which was quite evident by inspection of the text itself.  The first step in deciphering a Vigenere Cipher is to determine the keyword length.  This was done by finding the longest repeated words in the text, determining their distance apart, and finally taking the "GCD" of that distance.  The keyword then was determined to be 5.  The next step was to determine the shift distances or the keyword itself.  This was done by assume original words for encrypted text, measuring the distance of the ciphered text letter to the plaintext letter for repeated or assumed repeated words.  These shifts were determined to be: \[7,16,9,2,11\] or the keyword "krypt".

\hbox{}
\noindent
\textbf{Translated:}\\
Down, down, down. Would the fall NEVER come to an end! ’I wonder how
many miles I’ve fallen by this time?’ she said aloud. ’I must be getting
somewhere near the centre of the earth. Let me see: that would be four
thousand miles down, I think--’ (for, you see, Alice had learnt several
things of this sort in her lessons in the schoolroom, and though this
was not a VERY good opportunity for showing off her knowledge, as there
was no one to listen to her, still it was good practice to say it over)
’--yes, that’s about the right distance--but then I wonder what Latitude
or Longitude I’ve got to?’ (Alice had no idea what Latitude was, or
Longitude either, but thought they were nice grand words to say.)

Presently she began again. ’I wonder if I shall fall right THROUGH the
earth! How funny it’ll seem to come out among the people that walk with
their heads downward! The Antipathies, I think--’ (she was rather glad
there WAS no one listening, this time, as it didn’t sound at all the
right word) ’--but I shall have to ask them what the name of the country
is, you know. Please, Ma’am, is this New Zealand or Australia?’ (and
she tried to curtsey as she spoke--fancy CURTSEYING as you’re falling
through the air! Do you think you could manage it?) ’And what an
ignorant little girl she’ll think me for asking! No, it’ll never do to
ask: perhaps I shall see it written up somewhere.’

Down, down, down. There was nothing else to do, so Alice soon began
talking again. ’Dinah’ll miss me very much to-night, I should think!’
(Dinah was the cat.) ’I hope they’ll remember her saucer of milk at
tea-time. Dinah my dear! I wish you were down here with me! There are no
mice in the air, I’m afraid, but you might catch a bat, and that’s very
like a mouse, you know. But do cats eat bats, I wonder?’ And here Alice
began to get rather sleepy, and went on saying to herself, in a dreamy
sort of way, ’Do cats eat bats? Do cats eat bats?’ and sometimes, ’Do
bats eat cats?’ for, you see, as she couldn’t answer either question,
it didn’t much matter which way she put it. She felt that she was dozing
off, and had just begun to dream that she was walking hand in hand with
Dinah, and saying to her very earnestly, ’Now, Dinah, tell me the truth:
did you ever eat a bat?’ when suddenly, thump! thump! down she came upon
a heap of sticks and dry leaves, and the fall was over.

Alice was not a bit hurt, and she jumped up on to her feet in a moment:
she looked up, but it was all dark overhead; before her was another
long passage, and the White Rabbit was still in sight, hurrying down it.
There was not a moment to be lost: away went Alice like the wind, and
was just in time to hear it say, as it turned a corner, ’Oh my ears
and whiskers, how late it’s getting!’ She was close behind it when she
turned the corner, but the Rabbit was no longer to be seen: she found
herself in a long, low hall, which was lit up by a row of lamps hanging
from the roof.



\section{Matlab Code}
\textbf{Affine Decoder}
\begin{lstlisting}
%affine decoder
%fid=fopen('R:\git~1\ECE578\hw1.txt');
fid=fopen('~/Git/ECE578/hw1.txt');
txt=fread(fid);
%remove beginning txt
txt=txt(298:1968);

%F=a*m+b
%FOUND: a=7,b=13
%a=1:2:15;

txt_saved=txt;
for a=1:2:26
for b=0:25
    txt=txt_saved;
    for i=1:length(txt) 
        if (txt(i)>=65 && txt(i)<=65+25)
            %remove bias
            temp=txt(i);
            temp=temp-65;
            [~,a_i,~]=gcd(a,26);
            a_i=mod(a_i,26);
            temp=mod(round(a_i*(temp-b)),26);
            txt(i)=round(temp)+65;
        elseif(txt(i)>=97 && txt(i)<=97+25)
            %remove bias
            temp=txt(i);
            temp=temp-97;
            [~,a_i,~]=gcd(a,26);
            a_i=mod(a_i,26);
            temp=mod(round(a_i*(temp-b)),26);
            txt(i)=round(temp)+97;
        end       
    end
    %Look at sample output
    %char(txt(1:30))'
    %b
    %a
    %pause
end
end

%break
%converted txt
for a=7
for b=13
    txt=txt_saved;
    for i=1:length(txt) 
        if (txt(i)>=65 && txt(i)<=65+25)
            %remove bias
            temp=txt(i);
            temp=temp-65;
            [~,a_i,~]=gcd(a,26);
            a_i=mod(a_i,26);
            temp=mod(round(a_i*(temp-b)),26);
            txt(i)=round(temp)+65;
        elseif(txt(i)>=97 && txt(i)<=97+25)
            %remove bias
            temp=txt(i);
            temp=temp-97;
            [~,a_i,~]=gcd(a,26);
            a_i=mod(a_i,26);
            temp=mod(round(a_i*(temp-b)),26);
            txt(i)=round(temp)+97;
        end       
    end
    
end
end

txt_file=char(txt);
%save to file
fid2 = fopen('~/Git/ECE578/hw1_p1.txt','w');
fprintf(fid2,txt_file);
\end{lstlisting}

\hbox{}
\textbf{Vigenere Decoder}
\begin{lstlisting}
%letter shifter
fid=fopen('~/Git/ECE578/hw1.txt');
txt=fread(fid);
%remove beginning txt
txt=txt(1989:end);



shifts=[7 16 9 2 11];
txt(1)=mod((txt(1)-65)+shifts(2),26)+65;
txt(2)=mod((txt(2)-97)+shifts(3),26)+97;
txt(3)=mod((txt(3)-97)+shifts(4),26)+97;
txt(4)=mod((txt(4)-97)+shifts(5),26)+97;

j=1;
shift=0;
for i=1:length(txt)
    
    if (txt(i)>=65 && txt(i)<=65+25)||(txt(i)>=97 && txt(i)<=97+25)
        shift=shift+1;
    end
    if shift>=5 
    if (txt(i)>=65 && txt(i)<=65+25)
            %remove bias
            temp=txt(i);
            temp=temp-65;
            temp=mod(temp+shifts(j),26);
            txt(i)=round(temp)+65;
            shift=0;
        elseif(txt(i)>=97 && txt(i)<=97+25)
            %remove bias
            temp=txt(i);
            temp=temp-97;
            temp=mod(temp+shifts(j),26);
            txt(i)=round(temp)+97; 
            shift=0;
        end
    end
    
    
end
j=2;
%char(txt(1:30))'
shift=-1;
for i=1:length(txt)
    
    if (txt(i)>=65 && txt(i)<=65+25)||(txt(i)>=97 && txt(i)<=97+25)
        shift=shift+1;
    end
    if shift>=5 
    if (txt(i)>=65 && txt(i)<=65+25)
            %remove bias
            temp=txt(i);
            temp=temp-65;
            temp=mod(temp+shifts(j),26);
            txt(i)=round(temp)+65;
            shift=0;
        elseif(txt(i)>=97 && txt(i)<=97+25)
            %remove bias
            temp=txt(i);
            temp=temp-97;
            temp=mod(temp+shifts(j),26);
            txt(i)=round(temp)+97; 
            shift=0;
        end
    end
    
    
end
j=3;
%char(txt(1:30))'
shift=-2;
for i=1:length(txt)
    
    if (txt(i)>=65 && txt(i)<=65+25)||(txt(i)>=97 && txt(i)<=97+25)
        shift=shift+1;
    end
    if shift>=5 
    if (txt(i)>=65 && txt(i)<=65+25)
            %remove bias
            temp=txt(i);
            temp=temp-65;
            temp=mod(temp+shifts(j),26);
            txt(i)=round(temp)+65;
            shift=0;
        elseif(txt(i)>=97 && txt(i)<=97+25)
            %remove bias
            temp=txt(i);
            temp=temp-97;
            temp=mod(temp+shifts(j),26);
            txt(i)=round(temp)+97; 
            shift=0;
        end
    end
    
    
end
j=4;
%char(txt(1:30))'
shift=-3;

for i=1:length(txt)
    
    if (txt(i)>=65 && txt(i)<=65+25)||(txt(i)>=97 && txt(i)<=97+25)
        shift=shift+1;
    end
    if shift>=5 
    if (txt(i)>=65 && txt(i)<=65+25)
            %remove bias
            temp=txt(i);
            temp=temp-65;
            temp=mod(temp+shifts(j),26);
            txt(i)=round(temp)+65;
            shift=0;
        elseif(txt(i)>=97 && txt(i)<=97+25)
            %remove bias
            temp=txt(i);
            temp=temp-97;
            temp=mod(temp+shifts(j),26);
            txt(i)=round(temp)+97; 
            shift=0;
        end
    end
    
    
end
j=5;
%char(txt(1:30))'
shift=-4;

for i=1:length(txt)
    
    if (txt(i)>=65 && txt(i)<=65+25)||(txt(i)>=97 && txt(i)<=97+25)
        shift=shift+1;
    end
    if shift>=5 
    if (txt(i)>=65 && txt(i)<=65+25)
            %remove bias
            temp=txt(i);
            temp=temp-65;
            temp=mod(temp+shifts(j),26);
            txt(i)=round(temp)+65;
            shift=0;
        elseif(txt(i)>=97 && txt(i)<=97+25)
            %remove bias
            temp=txt(i);
            temp=temp-97;
            temp=mod(temp+shifts(j),26);
            txt(i)=round(temp)+97; 
            shift=0;
        end
    end
    
    
end

txt_file=char(txt);
%save to file
fid2 = fopen('~/Git/ECE578/hw1_p2.txt','w');
fprintf(fid2,txt_file);
\end{lstlisting}

\end{document}
