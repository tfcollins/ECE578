\documentclass[12pt,letterpaper]{article}

\usepackage{amsmath} % just math
\usepackage{amssymb} % allow blackboard bold (aka N,R,Q sets)
\usepackage{ulem}
\linespread{1.6}  % double spaces lines
\usepackage[left=1in,top=1in,right=1in,bottom=1in,nohead]{geometry}
\usepackage{enumerate}

\begin{document}

\linespread{1} % single spaces lines
\small \normalsize %% dumb, but have to do this for the prev to work
\begin{flushright}
Travis Collins \\
ECE578 Lecture 5 Notes \\
\today
\end{flushright}

\section{Multiple Encryption}
-Double encryption\\
Keyspace:\(\left|K\right|=2^k x 2^k=2^{2k}  \)\\
\[x => encrypt => y => encrypt =y'\]\\

\underline{Meet in the middle attack}\\
\[ x => encrypt(k(i)) => z => encrypt(k(j)) => y\]
\[e_{ki}(x)=z_{i}^1  e_{kj}^{-1}(y)=z_{j}^(2)\]

Input: (x',y'),(x",y")\\
Idea: Compute
\[z_{i}^1=e_{ki}(x)\]\\
\[z_{j}^2=e_{kij}^{-1}(y)\]\\
Problem: Find \(z_{i}^(1)=z_{j}^(2)\)\\

Procedure:\\
1.) Compute lookup table \((z_{i}^{(1)},k_{i}),i=1,2,...,2^k\)\\
Storage: \(2^k,(n+k) bits\)\\
2.) Sort according to \(z_{i}\) column (Done typically while building table in step 1) \\

Values from the first encryption:\\
\begin{table}
    \begin{tabular}{|l|l|}
        $ki$ & $zi(1)$ \\ 
        0  & 3     \\ 
        1  & 1     \\ 
        2  & 6     \\ 
        3  & 2     \\ 
        4  & 7     \\ 
        5  & 10    \\
    \end{tabular}
\end{table}

\begin{table}
    \begin{tabular}{|l|l|}
        $k$ & $zi(1)$ \\
        1  & 1     \\ 
        3  & 2     \\ 
        0  & 3     \\ 
        2  & 6     \\ 
        4  & 7     \\ 
        5  & 10    \\
    \end{tabular}
\end{table}

Use quick sort to look through table\\
Binary search: \(log2(2^k)=k \) (iterations), k=keylength\\


3.) Find matching \(z_{j}^{(2)}\)\\
a.) Compute \(e_{kj}^{-1}(y^1)=z_{j}^{(2)}\)\\
b.) If \[z_{j}^{(2)}\] is in lookup table, i.e. \(z_{i}^{(1)}=z_{j}^{(2)} => (k_{i},k_{j}) -> try (x",y")(x''',y''')\)\\
c.) If \((k_{i},k_{j})\) give matchin encryption return \((k_{i},k_{j})\) else goto 'a' try different \[k_{j}\]\\


Complexity:\\
Brute Force: \(2^{2k}\) encryptions (2x per iteration)\\
Meet in the middle attack: \(Time = 2^k(Lookup table i_values) + 2^k (online j_values)\)\\


Triple encrytion:\\
Attack on first encryption (1)\\
\(Time = 2^k+2^{2k}\)\\
\(Space = 2^k\)\\
Attack with second encryption (2)\\
\(Time = 2^{2k}+2^{k}\)\\
\(Space = 2^{2k}\)\\


Question: How many additional pairs (x",y"), (x''',y''').... etc should we test?\\

Assume in general we have an encryption system with 'l' subsequence encryptions\\

Step1.)In step 1 we found a keypair such that \(e_{k}...e_{kj}(e_{ki}(x'))=y' (l encryptions) \)\\
There are \(2^{lk}\) key combinations\\

How many possible values do I have for the cyphertext y' is \(2^n (n=blocksize)\)\\
One to one mapping x->y ( \(2^n\) possible outputs), \(2^{lk}/2^n number of mappings per ciphertext \)\\
Number of keys that are found that are incorrect \(2^{lk}/2^n-1\)\\

Step2.)  We now use the candiadate key from step 1 and check if \(e^l(x")=y"\)\\

If a random key is used, the likelyhood that \(e^l(x")=y"\) is \(1/2^n\)\\ 

If we check a third pair (x''',y'''), under the same random pair the probability will be:\(1/2^{2n}\)\\

If we check (t-1) additional pairs then the probability becomes \( 1/2^{(t-1)n}\)\\


3.) Since there are \(2^{lk}/2^n\) candidate keys in 1.) then probability that at least one fullfills all \(e^l(x')=y',e^l(x")=y",...,e^l(x'..')=y'..'\) is \[(Number of bad keys)2^{lk}/2^n*1/2^{(t-1)n}(prob of passing t-1 tests)=2^{lk-tn}\]\\


Example: Double encryption with DES\\
k=56,n=64,l=2\\
if t=1 (special case), \(Failure 1-1/2^{112}/2^{64}=1-1/2^48\)\\
if t=2 (special case), \(Failure 1/2^{16}\)\\
if t=3 (special case), \(Failure 1/2^{80}\)\\
if t=4 (special case), \(Failure 1/2^{144}\)\\






\end{document}
